\documentclass[11pt]{article}
\usepackage{amsmath}

\begin{document}

\section*{Rectangular Window}

\begin{equation*}
w_R(n) =
\begin{cases}
1, &-\frac{M-1}{2} \leq n \leq \frac{M-1}{2} \\
0, &\textit{otherwise}
\end{cases}
\end{equation*}

\begin{equation*}
W_R(\omega) = \frac{\sin(M\frac{\omega}{2})}{\sin(\frac{\omega}{2})} = M \cdot asinc_M(\omega)
\end{equation*}

\subsection*{Causal Case}

\begin{equation*}
W^c_R(\omega) = e ^ {-j\frac{M-1}{2}\omega} \cdot M \cdot asinc_M(\omega)
\end{equation*}

For zero-centered symmetric windows, we obtain real window transforms. Phase of $W_R(\omega)$ is 0 of $|\omega| < \frac{2\pi}{M}$, which is the size of mainlobe. $\pi$ occurs in side lobes.

\begin{itemize}
\item Main-lobe width = $\frac{4\pi}{M} rad$
\item Side-lobe level = 13dB down, roll-off 6dB/octave
\item Side-lobe width = $\Omega_M = \frac{2\pi}{M}$
\item Zero crossing at integer multiples of $\Omega_M$
\end{itemize}

The abruptness of window discontinuity in the time domain is also what determines the side-lobe roll-off rate.

\end{document}
